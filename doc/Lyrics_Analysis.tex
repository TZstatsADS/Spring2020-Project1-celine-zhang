\documentclass[]{article}
\usepackage{lmodern}
\usepackage{amssymb,amsmath}
\usepackage{ifxetex,ifluatex}
\usepackage{fixltx2e} % provides \textsubscript
\ifnum 0\ifxetex 1\fi\ifluatex 1\fi=0 % if pdftex
  \usepackage[T1]{fontenc}
  \usepackage[utf8]{inputenc}
\else % if luatex or xelatex
  \ifxetex
    \usepackage{mathspec}
  \else
    \usepackage{fontspec}
  \fi
  \defaultfontfeatures{Ligatures=TeX,Scale=MatchLowercase}
\fi
% use upquote if available, for straight quotes in verbatim environments
\IfFileExists{upquote.sty}{\usepackage{upquote}}{}
% use microtype if available
\IfFileExists{microtype.sty}{%
\usepackage{microtype}
\UseMicrotypeSet[protrusion]{basicmath} % disable protrusion for tt fonts
}{}
\usepackage[margin=1in]{geometry}
\usepackage{hyperref}
\hypersetup{unicode=true,
            pdftitle={Lyrics},
            pdfborder={0 0 0},
            breaklinks=true}
\urlstyle{same}  % don't use monospace font for urls
\usepackage{graphicx,grffile}
\makeatletter
\def\maxwidth{\ifdim\Gin@nat@width>\linewidth\linewidth\else\Gin@nat@width\fi}
\def\maxheight{\ifdim\Gin@nat@height>\textheight\textheight\else\Gin@nat@height\fi}
\makeatother
% Scale images if necessary, so that they will not overflow the page
% margins by default, and it is still possible to overwrite the defaults
% using explicit options in \includegraphics[width, height, ...]{}
\setkeys{Gin}{width=\maxwidth,height=\maxheight,keepaspectratio}
\IfFileExists{parskip.sty}{%
\usepackage{parskip}
}{% else
\setlength{\parindent}{0pt}
\setlength{\parskip}{6pt plus 2pt minus 1pt}
}
\setlength{\emergencystretch}{3em}  % prevent overfull lines
\providecommand{\tightlist}{%
  \setlength{\itemsep}{0pt}\setlength{\parskip}{0pt}}
\setcounter{secnumdepth}{0}
% Redefines (sub)paragraphs to behave more like sections
\ifx\paragraph\undefined\else
\let\oldparagraph\paragraph
\renewcommand{\paragraph}[1]{\oldparagraph{#1}\mbox{}}
\fi
\ifx\subparagraph\undefined\else
\let\oldsubparagraph\subparagraph
\renewcommand{\subparagraph}[1]{\oldsubparagraph{#1}\mbox{}}
\fi

%%% Use protect on footnotes to avoid problems with footnotes in titles
\let\rmarkdownfootnote\footnote%
\def\footnote{\protect\rmarkdownfootnote}

%%% Change title format to be more compact
\usepackage{titling}

% Create subtitle command for use in maketitle
\providecommand{\subtitle}[1]{
  \posttitle{
    \begin{center}\large#1\end{center}
    }
}

\setlength{\droptitle}{-2em}

  \title{Lyrics}
    \pretitle{\vspace{\droptitle}\centering\huge}
  \posttitle{\par}
    \author{}
    \preauthor{}\postauthor{}
    \date{}
    \predate{}\postdate{}
  
\usepackage{booktabs}
\usepackage{longtable}
\usepackage{array}
\usepackage{multirow}
\usepackage{wrapfig}
\usepackage{float}
\usepackage{colortbl}
\usepackage{pdflscape}
\usepackage{tabu}
\usepackage{threeparttable}
\usepackage{threeparttablex}
\usepackage[normalem]{ulem}
\usepackage{makecell}
\usepackage{xcolor}

\begin{document}
\maketitle

Nowadays, music is a big part of people's life. Lyrics is considered as
the soul of music. There are so many kinds of music, like Rock and
Metal, but, as for me, I am not familiar with all kinds of genres. So,
in this project, I am trying exploring the lyrics and understanding the
underlying things behind the music's genres.

\hypertarget{exploring-the-relationship-between-length-and-genres}{%
\subsection{Exploring the relationship between length and
genres}\label{exploring-the-relationship-between-length-and-genres}}

\includegraphics{Lyrics_Analysis_files/figure-latex/unnamed-chunk-3-1.pdf}
\includegraphics{Lyrics_Analysis_files/figure-latex/unnamed-chunk-3-2.pdf}
\includegraphics{Lyrics_Analysis_files/figure-latex/unnamed-chunk-3-3.pdf}

From the histogram, we notice that the number of words in hip-hop music
is commonly more than that in other genres. Also, in the boxplot, we can
see that hip-hop contains the most words and the eletronic has the
least. From the violin-plot, it is shown that, the number of words in
the genres, such as country, jazz and rock music, is kind of consistent
as time goes by. The Eletronic music has the longest range in the number
of words. For next step, I am trying to figure out the relationship
between the distribution of number of lyrics and the decades, so I pick
the Metal Music as the example.

\includegraphics{Lyrics_Analysis_files/figure-latex/unnamed-chunk-4-1.pdf}
According to the ridges plot, there is evident that the length of lyrics
is increasing and the distribution became more concentrated as the year
goes by.

\hypertarget{the-analysis-of-content-in-lyrics}{%
\subsection{The analysis of content in
lyrics}\label{the-analysis-of-content-in-lyrics}}

\includegraphics{Lyrics_Analysis_files/figure-latex/unnamed-chunk-5-1.pdf}
As we can see, except for Metal, love is used the most frequently by all
the genres, and word ``love'' is mentioned over 80000 times, which is
the highest in all genres. However, in metal music, the terms like
``life'' and ``time'', are mentioned more than the term ``love''. I
guess this is resulted by the core of metal music, the aggressive
attitude about life rather than the joy of the world. We also can
conclude this from the word cloud of metal music's lyrics.

\includegraphics{Lyrics_Analysis_files/figure-latex/unnamed-chunk-6-1.pdf}

This wordcloud contains the top 200 frequently used terms in metal
music, and we can observe that most of terms express negative feelings
like die, pain and lie. Hence, I guess the major emotion of metal music
is negative. In order to explore the emotion underlying the metal music,
we use the sentiment analysis.

\hypertarget{semtiment-analysis}{%
\subsection{Semtiment analysis}\label{semtiment-analysis}}

\includegraphics{Lyrics_Analysis_files/figure-latex/unnamed-chunk-7-1.pdf}
As we can see, the top 5 major feeling of the metal music is negative,
positive, fear, sadness and anger, and four of five suggest the darkness
of life. Hence, we can infer that the reason why metal music is less
often to mention the term ``love'' is that its major emotion is
negative. We also can use the sentiment analysis on all the genres to
prove that.

\begin{verbatim}
## [1] "joy"      "positive"
\end{verbatim}

\includegraphics{Lyrics_Analysis_files/figure-latex/unnamed-chunk-8-1.pdf}
As the result shown in the sentiment analysis, the term `love' is
corresponding to joy and positive emotion, and also considering what the
heatmap is shown,compared to other genres, the metal music has more
preference on the negative emotion rather than the joyful and positive
emotions. Therefore, the metal music relatively rarely mentions the term
love.

\hypertarget{the-distribution-of-metal-artists}{%
\subsection{The distribution of metal
artists}\label{the-distribution-of-metal-artists}}

I wanna find out the distribution of metal music as years goes by
\includegraphics{Lyrics_Analysis_files/figure-latex/unnamed-chunk-9-1.pdf}
\includegraphics{Lyrics_Analysis_files/figure-latex/unnamed-chunk-9-2.pdf}
\includegraphics{Lyrics_Analysis_files/figure-latex/unnamed-chunk-9-3.pdf}
\includegraphics{Lyrics_Analysis_files/figure-latex/unnamed-chunk-9-4.pdf}
\includegraphics{Lyrics_Analysis_files/figure-latex/unnamed-chunk-9-5.pdf}
As we can see from the pie chart, the first metal music was created in
Sweden. And during 1985 to 1995, most of metal artist were still from
Sweden, with a few of them came from the United States. During 1995 to
2005, the metal music was spread to other European coutries, such as
Finland and England. After 2005, the metal music is spread around
Europe, North America.

\hypertarget{conclusion}{%
\subsection{conclusion}\label{conclusion}}

1 Hip-Hop commonly contains the longer words compared to other genres.
The number of words in metal music seems to increase as time goes by.

2 The reason why the metal music is less often to mention the term
``love'' is that its major emotion is more about the darkness and
negative feelings about the world.

3 Most of metal artists are from Sweden.


\end{document}
